\documentclass{article}

\usepackage[utf8]{inputenc} % Este pacote serve para acentuação
\usepackage[brazil]{babel} % Este pacote coloca os nomes em pt-br
\usepackage{indentfirst} % Este pacote aplica indentação
\usepackage[a4paper, left=3cm, right=2cm, top=3cm, bottom=2cm]{geometry} % Este pacote altera a margem do documento
\usepackage{graphicx} % Este pacote permite adicionar figuras
\usepackage{float} % Força o posicionamento da figura
\usepackage{fancyhdr}

\begin{document}
\begin{titlepage}
	\begin{figure}[H]
		\centering
		\includegraphics[width=0.1\linewidth]{Figuras/logofei}
		\label{fig:logofei}
	\end{figure}
	
	\begin{center}
		{\large \textbf{Centro Universitário FEI} }
	\end{center}
	
	\vspace*{3cm}
	
	\begin{center}
		{\Large {\bf{ Projeto de CA-2330 \\ 
					\vspace*{0.5cm}{\large Relatório - Parte 1}}}}
	\end{center}\vspace*{5cm}
	
	\begin{center}
		Grupo Nº 11 \vspace*{0.2cm} \\ João Pedro Rosa Cezarino - R.A: 22.120.021-5  \\ Lucca Bonsi Guarreschi - R.A: 22.120.016-5  \\ Vitor Martins Oliveira - R.A: 22.120.067-8
	\end{center}
	
	\vspace*{7cm}
	\begin{center} \textbf{São Bernardo do Campo \\ 2020}\end{center}
	
\end{titlepage}
	\newpage
	\setcounter{page}{1} % Começa a contar as páginas novamente
	\pagenumbering{arabic} % altera para algarismo arábico
	
	\section{{\Large \underline{O que são Grafos?}}}
		Resumidamente, Grafos podem ser entendidos como um conjunto de pontos, chamados vértices, e outro de pares de pontos, chamados arestas.Cada aresta liga um par de pontos (extremidades) que a determina.Usualmente a representação é feita por meio da junção de arestas e vértices em um plano.
		
\begin{figure}[H]
	\centering
	\includegraphics[width=0.2\linewidth]{"Figuras/Grafos -1"}
	\label{fig:grafo1}
\end{figure}

		Os grafos permitem a modelagem de situações como: redes de computadores, de comunicação, a Web, arvores genealógicas entre outras diversas aplicações.\vspace*{0.5cm}
		
	\section{{\Large \underline{Grafos Simples}}}
		Os Grafos simples são aqueles em que não se tem laços, nem mais de uma aresta ligando dois vértices.
		
\begin{figure}[H]
	\centering
	\includegraphics[width=0.2\linewidth]{"Figuras/Grafos -2"}
	\label{fig:grafos2}
\end{figure}
		
		Os Grafos que possuem arestas múltiplas e/ou laços são chamadas Multigrafos.
		
\begin{figure}[H]
	\centering
	\includegraphics[width=0.2\linewidth]{"Figuras/Grafo -4"}
	\label{fig:grafo4}
\end{figure}\vspace*{0.5cm}

	\section{{\Large \underline{Laços}}}
		Um laço é uma aresta que conecta um vértice a ele mesmo.
	
\begin{figure}[H]
	\centering
	\includegraphics[width=0.2\linewidth]{"Figuras/laco"}
	\label{fig:laco}
\end{figure}
	\newpage	
	\section{{\Large \underline{Arestas Múltiplas}}}
		Arestas Múltiplas são definidas como arestas que possuem os mesmos vértices como extremidades.
		
\begin{figure}[H]
	\centering
	\includegraphics[width=0.2\linewidth]{"Figuras/arestasm"}
	\label{fig:arestasm}
\end{figure}\vspace*{0.5cm}

	\section{{\Large \underline{Descrição do Programa}}}
	
	Inicialmente o arquivo onde se encontra a matriz de adjacência "A.txt" é aberto e a leitura de cada linha da matriz é realizada, com o objetivo de retirar os espaços ("$\backslash$n") e transformar o conteúdo em números inteiros (do tipo "int"). Após esse processo, o programa verifica e retira (se houver a presença) os espaços em branco desnecessários na matriz.
	
\begin{figure}[H]
	\centering
	\includegraphics[width=0.2\linewidth]{"Figuras/Entrada - Matriz"}
	\label{fig:entradamatriz}
\end{figure}

	 Os laços de repetição exercem a tarefa de percorrer todas as linhas e colunas da matriz do grafo, localizando assim, seus laços e arestas e armazenando-os. O programa encontra um laço quando a coluna("$i$") e a linha("$j$") forem iguais e o número encontrado nelas for maior que 0. Já para as Arestas Múltiplas, o programa as identifica quando o número entre a linha e a coluna for superior à 1. Por fim, a classificação do grafo é impressa no terminal (se é simples ou não) e os laços e arestas múltiplas são apresentados ao usuário, assim como suas respectivas posições, como forma de comprovar a resposta.
	 
\begin{figure}[H]
	\centering
	\includegraphics[width=0.5\linewidth]{"Figuras/Saida - Programa"}
	\label{fig:saidaprograma}
\end{figure}
	 
	 
\end{document}
