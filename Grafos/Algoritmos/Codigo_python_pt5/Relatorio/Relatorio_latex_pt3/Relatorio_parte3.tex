\documentclass{article}

\usepackage[utf8]{inputenc} % Este pacote serve para acentuação
\usepackage[brazil]{babel} % Este pacote coloca os nomes em pt-br
\usepackage{indentfirst} % Este pacote aplica indentação
\usepackage[a4paper, left=3cm, right=2cm, top=3cm, bottom=2cm]{geometry} % Este pacote altera a margem do documento
\usepackage{graphicx} % Este pacote permite adicionar figuras
\usepackage{float} % Força o posicionamento da figura
\usepackage{fancyhdr}
\usepackage{amsmath}
\usepackage{amsfonts}
\usepackage{amssymb}
\usepackage{caption}
%\usepackage{subfigure}

\begin{document}
\begin{titlepage}
	\begin{figure}[H]
		\centering
		\includegraphics[width=0.1\linewidth]{Figuras/logofei}
		\label{fig:logofei}
	\end{figure}
	
	\begin{center}
		{\large \textbf{Centro Universitário FEI} }
	\end{center}
	
	\vspace*{3cm}
	
	\begin{center}
		{\Large {\bf{ Projeto de CA-2330 \\ 
					\vspace*{0.5cm}{\large Relatório - Parte 3}}}}
	\end{center}\vspace*{5cm}
	
	\begin{center}
		Grupo Nº 11 \vspace*{0.2cm} \\ João Pedro Rosa Cezarino - R.A: 22.120.021-5  \\ Lucca Bonsi Guarreschi - R.A: 22.120.016-5  \\ Vitor Martins Oliveira - R.A: 22.120.067-8
	\end{center}
	
	\vspace*{7cm}
	\begin{center} \textbf{São Bernardo do Campo \\ 2020}\end{center}
	
\end{titlepage}
	\newpage
	\setcounter{page}{1} % Começa a contar as páginas novamente
	\pagenumbering{arabic} % altera para algarismo arábico
	
	\section{{\Large \underline{Grafos Bipartidos}}}
		Um grafo é chamado bipartido quando seu conjunto de vértices $V$ puder ser dividido em dois subconjuntos $V$1 e $V2$, tais que cada aresta de $G$ une um vértice de $V$1 a outro de $V2$.\vspace*{0.2cm}
		
		Cada aresta de $G$ possui um extremo em $X$ e outro em $Y$ (Os extremos deverão estar em conjuntos diferentes).
		
\begin{figure}[H]
	\centering
	\includegraphics[width=0.2\linewidth]{"Figuras/img1"}
	\label{fig:img1}
	\caption{$X={1,2}$ e $Y={4,3}$}
\end{figure}
	Para demonstrar que um grafo $G$ é bipartido basta mostrar uma bipartição de $VG$ na qual os extremos estejam em conjuntos diferentes. 

	\section{{\Large \underline{Grafos Bipartidos Completos}}}
		Um grafo é dito bipartido completo se for completo e se todo vértice de $X$ é adjacente a todo vértice de $Y$. Portanto, $X$ e $Y$ são independentes. Usualmente um grafo bipartido completo é denotado por $K_{m,n}$, onde $m$ é o número de vértices em $V1$ e $n$ é o número de vértices em $V2$.
		
\begin{figure}[H]
	\centering
	\includegraphics[width=0.5\linewidth]{"Figuras/img2"}
	\label{fig:img2}
	\caption{}
\end{figure}\vspace*{0.3cm}
	Um grafo bipartido completo $Km,n$ tem $m*n$ arestas
	
\begin{figure}[H]
	\centering
	\includegraphics[width=0.2\linewidth]{"Figuras/img3"}
	\label{fig:img3}
	\caption{$ K_{2,2} $ / 4 Arestas}
\end{figure}\vspace*{0.3cm}
\begin{figure}[H]
	\centering
	\includegraphics[width=0.2\linewidth]{"Figuras/img4"}
	\label{fig:img4}
	\caption{$ K_{3,3} $ / 9 Arestas}
\end{figure}\vspace*{0.3cm}
	
	O grafo $G6$ é um $ K_{3,3} $, ou seja, um grafo bipartido completo que contém dois conjuntos de 3 vértices cada. Ele é completo pois todos os vértices de um conjunto estão ligados a todos os vértices do outro conjunto.

\begin{figure}[H]
	\centering
	\includegraphics[width=0.5\linewidth]{"Figuras/img5"}
	\label{fig:img5}
	\caption{}
\end{figure}\vspace*{0.2cm}

	\section{{\Large \underline{Descrição do Programa}}}\vspace*{0.3cm}
	
	Inicialmente o arquivo onde se encontra a matriz de adjacência "A.txt" é aberto e a leitura de cada linha da matriz é realizada, com o objetivo de retirar os espaços ("$\backslash$n") e transformar o conteúdo em números inteiros (do tipo "int"). Após esse processo, o programa verifica e retira (se houver a presença) os espaços em branco desnecessários na matriz.
	
	\begin{figure}[H]
		\centering
		\includegraphics[width=0.6\linewidth]{"Figuras/Print1"}
		\label{fig:print1}
		\caption{}
	\end{figure}\vspace*{0.3cm}
	\newpage

		Após a abertura do arquivo, duas variáveis e duas listas são inicializadas. A função "adjacentes(y)" verifica quais vértices do grafo são adjacentes entre si e retorna "True" caso eles sejam adjacentes. A matriz então é percorrida com o objetivo de checar se os vértices podem ser divididos em dois grupos distintos.

	\begin{figure}[H]
		\centering
		\includegraphics[width=0.6\linewidth]{"Figuras/Print2"}
		\label{fig:print2}
		\caption{}
	\end{figure}\vspace*{0.3cm}
	
		Após as instruções acima, o programa passa a verificar se o grafo é bipartido ou não. Caso ele seja, uma bipartição será exibida no terminal.
	
	\begin{figure}[H]
		\centering
		\includegraphics[width=0.6\linewidth]{"Figuras/Print3"}
		\label{fig:print3}
		\caption{}
	\end{figure}\vspace*{0.3cm}

		Após a verificação da bipartição, o programa verifica se o grafo pode ser bipartido completo ou não.
	 
	 \begin{figure}[H]
	 	\centering
	 	\includegraphics[width=0.6\linewidth]{"Figuras/Print4"}
	 	\label{fig:print4}
	 	\caption{}
	 \end{figure}\vspace*{0.3cm}
 	\newpage
 
 		Por fim, os resultados são impressos no terminal, como nos exemplos abaixo:
 	
 	\begin{figure}[H]
 		\centering
 		\includegraphics[width=0.6\linewidth]{"Figuras/Print5"}
 		\label{fig:print5}
 		\caption{}
 	\end{figure}\vspace*{0.3cm}
 
	 \begin{figure}[H]
	 	\centering
	 	\includegraphics[width=0.6\linewidth]{"Figuras/Print6"}
	 	\label{fig:print6}
	 	\caption{}
	 \end{figure}\vspace*{0.3cm}

	\begin{figure}[H]
		\centering
		\includegraphics[width=0.6\linewidth]{"Figuras/Print7"}
		\label{fig:print7}
		\caption{}
	\end{figure}\vspace*{0.3cm}
 
 
\end{document}