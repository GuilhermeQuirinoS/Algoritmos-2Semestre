\documentclass{article}

\usepackage[utf8]{inputenc} % Este pacote serve para acentuação
\usepackage[brazil]{babel} % Este pacote coloca os nomes em pt-br
\usepackage{indentfirst} % Este pacote aplica indentação
\usepackage[a4paper, left=3cm, right=2cm, top=3cm, bottom=2cm]{geometry} % Este pacote altera a margem do documento
\usepackage{graphicx} % Este pacote permite adicionar figuras
\usepackage{float} % Força o posicionamento da figura
\usepackage{fancyhdr}
\usepackage{amsmath}
\usepackage{amsfonts}
\usepackage{amssymb}
\usepackage{caption}
%\usepackage{subfigure}

\begin{document}
\begin{titlepage}
	\begin{figure}[H]
		\centering
		\includegraphics[width=0.1\linewidth]{Figuras/logofei}
		\label{fig:logofei}
	\end{figure}
	
	\begin{center}
		{\large \textbf{Centro Universitário FEI} }
	\end{center}
	
	\vspace*{3cm}
	
	\begin{center}
		{\Large {\bf{ Projeto de CA-2330 \\ 
					\vspace*{0.5cm}{\large Relatório - Parte 5}}}}
	\end{center}\vspace*{5cm}
	
	\begin{center}
		Grupo Nº 11 \vspace*{0.2cm} \\ João Pedro Rosa Cezarino - R.A: 22.120.021-5  \\ Lucca Bonsi Guarreschi - R.A: 22.120.016-5  \\ Vitor Martins Oliveira - R.A: 22.120.067-8
	\end{center}
	
	\vspace*{7cm}
	\begin{center} \textbf{São Bernardo do Campo \\ 2020}\end{center}
	
\end{titlepage}
	\newpage
	\setcounter{page}{1} % Começa a contar as páginas novamente
	\pagenumbering{arabic} % altera para algarismo arábico
	
	\section{{\Large \underline{Grafos e Trilhas de Euler}}}
	Um grafo G é dito ser euleriano se há uma trilha em G que contenha todas as suas arestas. Esta trilha é dita ser uma trilha euleriana. O grafo da figura abaixo por exemplo, é euleriano já que ele contém a trilha: $(u1, u2, u3, u4, u5, u3, u1, u6, u2, u7, u3, u6, u7, u1)$, que é euleriana.\vspace*{0.2cm}
		
\begin{figure}[H]
	\centering
	\includegraphics[width=0.4\linewidth]{"Figuras/img1"}
	\label{fig:img1}
	\caption{}
\end{figure}

	Um Grafo conexo G é Euleriano se cada vértice de G possui grau par, ou seja, se o grafo é euleriano todos os vértices têm grau par, e além disto, se todos os vértices do grafo têm grau par então o grafo é euleriano.
		
\begin{figure}[H]
	\centering
	\includegraphics[width=0.3\linewidth]{"Figuras/img2"}
	\label{fig:img2}
	\caption{}
\end{figure}\vspace*{0.3cm}
	Outra condição que provê uma solução simples para determinar se um grafo é euleriano é a de que um grafo M é euleriano se, e somente se, M é conexo.
	
\begin{figure}[H]
	\centering
	\includegraphics[width=0.5\linewidth]{"Figuras/img3"}
	\label{fig:img3}
	\caption{}
\end{figure}\vspace*{0.2cm}
		
		\section{{\Large \underline{Descrição do Programa}}}\vspace*{0.3cm}
		
		Inicialmente o arquivo onde se encontra a matriz de adjacência "A.txt" é aberto e a leitura de cada linha da matriz é realizada, com o objetivo de retirar os espaços ("$\backslash$n") e transformar o conteúdo em números inteiros (do tipo "int"). Após esse processo, o programa verifica e retira (se houver a presença) os espaços em branco desnecessários na matriz.\\
		
		Então, Transforma-se a lista extraída do arquivo que contém a matriz de adjacência em um array que pode ser interpretado pela biblioteca $Numpy$ e dessa forma podemos incorporar esta matriz aos módulos da biblioteca $NetworkX$.
		
		\begin{figure}[H]
			\centering
			\includegraphics[width=0.9\linewidth]{"Figuras/print1"}
			\label{fig:print1}
			\caption{}
		\end{figure}\vspace*{0.2cm}

 		Logo após a execução anterior, utiliza-se a função \textit{is\_eulerian} da biblioteca $NetworkX$, que retorna $TRUE$ caso o grafo for Euleriano e $FALSE$ se não for. Portanto, se a função retornar $TRUE$ o grafo em questão é Euleriano, caso contrário, o grafo não é. A função \textit{has\_eulerian\_path} da biblioteca $NetworkX$ é responsável por retornar $TRUE$ caso o grafo possua um trilha de Euler e $FALSE$ se não possuir. Se o grafo possuir uma trilha Euleriana ela será impressa no terminal junto com as outras informações.
 		
 		\begin{figure}[H]
 			\centering
 			\includegraphics[width=0.7\linewidth]{"Figuras/print2"}
 			\label{fig:print2}
 			\caption{}
 		\end{figure}\vspace*{0.2cm}
 	
 		Por fim, obtém-se os resultados impressos no terminal e uma representação gráfica do grafo que foi submetido ao programa. É importante ressaltar que as linhas impressas na cor vermelha não se tratam de um erro e sim um aviso de um dos módulos utilizados no programa, quanto à imagem do grafo gerada. Tal aviso não interfere na imagem gerada e nos resultados finais do código.
 		
 		\begin{figure}[H]
 			\centering
 			\includegraphics[width=0.9\linewidth]{"Figuras/saida"}
 			\label{fig:saida}
 			\caption{}
 		\end{figure}\vspace*{0.2cm}
 		
 		\textbf{\underline{Atenção}}: Para o correto funcionamento do programa as bibliotecas \textit{Matplotlib}, \textit{NetworkX} e \textit{Numpy} devem estar instaladas no sistema. Para instalá-las, execute os comandos abaixo no terminal da sua IDE ou na linha de comando do sistema:
 		
 		\begin{center}
 			pip install Matplotlib\\
 			pip install NetworkX\\
 			pip install Numpy
 		\end{center}
 		
 		Também é importante lembrar que para instalar as bibliotecas acima, o instalador \textit{pip} deve estar instalado no sistema em que o programa será executado. Para instalá-lo, deve-se seguir o passo a passo da documentação que está presente no link abaixo: 
 		\begin{center}
 			\textit{https://pip.pypa.io/en/stable/installing/}
 		\end{center}\vspace*{0.2cm}
 	
 			A biblioteca \textbf{\textit{NetworkX}} é um pacote Python utilizado na criação, manipulação e estudo das estruturas, da dinâmica e das funções de redes complexas e de grafos.\\
 		
 		A biblioteca \textbf{\textit{Matplotlib}} tem a função de criar representações gráficas estáticas, animadas e interativas em Python.\\
 		
 		\textbf{\textit{NumPy}} é uma biblioteca para Python, que adiciona suporte para matrizes e arrays multidimensionais grandes, junto com uma grande coleção de funções matemáticas de alto nível para operar estes arrays.
 		
\end{document}